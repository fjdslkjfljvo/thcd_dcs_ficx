% \documentclass{beamer}
% \usepackage{beamerthemesplit}
% % \usetheme{Marburg}
% \usecolortheme{orchid}

\documentclass[9pt]{beamer}

\usetheme{Singapore}

\usepackage{amsfonts,amsmath,amssymb,amsthm,graphicx}
\usepackage{changepage}
\usepackage{enumerate}
\usepackage{algorithm}
\usepackage{algorithmic}
\usepackage{natbib}
\usepackage{nicefrac}
\usepackage{color}
\usepackage{algorithm}
\usepackage{algorithmic}
\usepackage{setspace}
\usepackage{graphicx}
\usepackage{amsthm,multirow,commath}
\usepackage{xcolor}
\usepackage{subcaption} % Needed for subfigure captions


\usepackage{sansmathaccent}
\pdfmapfile{+sansmathaccent.map}

%\newtheorem{theorem}{Theorem}
%\newtheorem{corollary}{Corollary}
%\newtheorem{newthm}{Theorem}
%\newtheorem{newdefn}{Definition}
\newtheorem{observation}{Observation}
%\newtheorem{example}{Example}
%\newtheorem{newclm}{Claim}
%\newtheorem{newconjecture}{Conjecture}
%\newtheorem{definition}{Definition}
%\newtheorem{lemma}{Lemma}
%\newtheorem{problem}{Problem}
\newtheorem{procedure}{Procedure}

\newcommand{\s}{\mathcal{S}}
\renewcommand{\i}{\mathcal{I}}
\renewcommand{\k}{\mathcal{K}}
\newcommand{\natr}{\mathcal{N}}
\newcommand{\udl}[1]{{\color{blue} #1}}




%----------------------------------------------------------------------------------------
%	TITLE PAGE
%----------------------------------------------------------------------------------------




\title[]{Non-linear model with heterogeneous features} % The short title appears at the bottom of every slide, the full title is only on the title page


\author{
{ \footnotesize
Mentor: Logan\\
Committee: Shengming, Lu
}
\\
\vspace{5mm}
Zhihao Jiang
}

\date{August 9, 2024}



\begin{document}

\frame{\titlepage}






\begin{frame}{Overview}


Given market data and 200+ features.
\begin{itemize}
    \item from 2021 to 2024-06, every second, 36M datapoints
    \item mid price, hbas
    \item linear features: s.trade, emd\_spread, \emph{etc.}
    \item non-linear features: hour, volume, \emph{etc.}
\end{itemize}

Objective: predict average forward return in 1-2 hours.
\begin{itemize}
    \item $\log\left(\frac{\text{average mid price 1-2 hours later}}{\text{current mid price}}\right)$
\end{itemize}

Contribution: feature selection, develop a tree-based model (XGBoost)



\end{frame}


\begin{frame}{Data Preprocessing}

\begin{itemize}
    \item Missing values. Forward fill prices.

    \item Subsample every 5 seconds.

    \item Data Splitting
    \begin{itemize}
        \item training: 2021-01 to 2022-03, 4 million, $58\%$
        \item validation: 2023-08 to 2024-03-15, 2 million, $30\%$
        \item testing: 2024-03-15 to 2024-06, 0.83 million, $12\%$
    \end{itemize}

    \item Standardization. $x\leftarrow \frac{x-x.mean}{x.std}$.

    \item Create new features, stats of existed features, combinations.
\end{itemize}


\end{frame}

\begin{frame}{First attempts}

\begin{itemize}
    \item Train an XGBoost model using all features
    \item Sort features by importance, and select first $k$ features.
    \item Forward selection.
\end{itemize}


\end{frame}


\begin{frame}{Only linear features: Lasso}


\begin{itemize}
    \item Remove non-linear features, such as volume, hour.
    \item Train using training set, choose lasso parameter $\alpha$ that maximizes $r^2$ on validation set.
\end{itemize}

\begin{figure}
    \centering
    \includegraphics[width=0.5\textwidth]{fig/lasso_r2.png}
    %\caption{lasso $r^2$}
\end{figure}


\begin{table}[h!]
\centering
\begin{tabular}{c|c|c|c}
\hline
 & \textbf{training} & \textbf{validation} & \textbf{testing} \\ \hline
$r^2$      & 0.44\%      & 0.29\%      & 0.60\%    \\ \hline
\end{tabular}
\end{table}


\end{frame}

\begin{frame}{ Only linear features: Select features by t-values}

\begin{itemize}
    \item Fit linear regression using all linear features.
    \item Calculate $t$-values of each feature. $t_i\propto \nicefrac{\beta_i}{SE(\beta_i)}$
    \item Remove highly correlated features.
    \item Sort features by t-values in linear regression, from high to low.
    \item Train a ridge regression model using first $k$ features.
\end{itemize}
\begin{figure}
    \centering
    \includegraphics[width=0.5\textwidth]{fig/ridge_r2.png}
    %\caption{t-value $r^2$}
\end{figure}


\begin{table}[h!]
\centering
\begin{tabular}{c|c|c|c}
\hline
 & \textbf{training} & \textbf{validation} & \textbf{testing} \\ \hline
$r^2$      & 0.39\%      & 0.35\%     & 0.60\%      \\ \hline
\end{tabular}
\end{table}


\end{frame}


\begin{frame}{Scale prediction by hour}



\begin{itemize}
	\item label $y$, prediction $\alpha$
	\item rescale prediction to be $c\cdot \alpha$, $c=\frac{\alpha^T y}{\alpha^T \alpha}$ maximizes $r^2$
    \item For each hour $i$, calculate the optimal $c_i$ for scaling prediction in this hour
\end{itemize}

\begin{figure}
    \centering
    \includegraphics[width=0.5\textwidth]{fig/hour_scale.png}
    %\caption{scales}
\end{figure}



\end{frame}


\begin{frame}{Scale prediction by hour}

\begin{itemize}
    \item final scaling factor $c^{*}_i = 1.0+a\cdot (c_i-1.0)^{b}$, where $a,b$ maximizes $r^2$ on the validation set
\end{itemize}

\begin{figure}
    \centering
    \includegraphics[width=0.5\textwidth]{fig/hour_full.png}
    %\caption{scales}
\end{figure}

\begin{table}[h!]
\centering
\begin{tabular}{c|c|c|c}
\hline
$r^2$ & \textbf{training} & \textbf{validation} & \textbf{testing} \\ \hline
lasso before scaling      & 0.44\%      & 0.29\%      & 0.60\%      \\ \hline
lasso after scaling      & 0.81\%      & 0.48\%      & 0.90\%      \\ \hline
%tree before scaling      & 0.81\%      & 0.39\%      & 0.83\%      \\ \hline
%tree after scaling      & 1.08\%      & 0.43\%      & 0.96\%      \\ \hline
\end{tabular}
\end{table}



\end{frame}




\begin{frame}{Tree-based model: XGBoost}


\begin{itemize}
    \item Using features selected in linear regression (both lasso and t-values)
    \item Linear regression predictions as features
    \item Removing original linear features make it better
    \item Add non-linear features by feature importance
    \begin{itemize}
        \item ave\_volume\_2h, ave\_sell\_volume\_2h, return\_from\_high\_6h, return\_from\_low\_30min, ave\_volume\_diff\_2h, hour
    \end{itemize}
\end{itemize}



\end{frame}




\begin{frame}{Simple tree}


\begin{figure}
    \centering
    \includegraphics[width=1.0\textwidth]{fig/simple_tree.png}
    %\caption{scales}
\end{figure}

\begin{table}[h!]
\centering
\begin{tabular}{c|c|c|c}
\hline
 & \textbf{training} & \textbf{validation} & \textbf{testing} \\ \hline
$r^2$      & 0.58\%      & 0.34\%      & 0.32\%      \\ \hline
\end{tabular}
\end{table}



\end{frame}



\begin{frame}{Hyperparameter tune}


\begin{figure}[h!]
    \centering
    \begin{subfigure}[b]{0.475\textwidth}
        \centering
        \includegraphics[width=\textwidth]{fig/tune_lambda.png}
        %\caption{Caption for Figure 1}
        \label{fig:figure1}
    \end{subfigure}
    \begin{subfigure}[b]{0.475\textwidth}
        \centering
        \includegraphics[width=\textwidth]{fig/tune_subsample.png}
        %\caption{Caption for Figure 2}
        \label{fig:figure2}
    \end{subfigure}
    %\caption{Main caption for the three figures.}
    \label{fig:three_figures}
\end{figure}



\begin{table}[h!]
\centering
\begin{tabular}{c|c|c|c}
\hline
$r^2$ & \textbf{training} & \textbf{validation} & \textbf{testing} \\ \hline
tree      & 0.81\%      & 0.39\%      & 0.83\%      \\ \hline
linear     & 0.44\%      & 0.29\%      & 0.60\%      \\ \hline
\end{tabular}
\end{table}



\end{frame}



\begin{frame}{Hyperparameter tune}



\begin{figure}[h!]
    \centering
    \begin{subfigure}[b]{0.475\textwidth}
        \centering
        \includegraphics[width=\textwidth]{fig/tune_lambda.png}
        %\caption{Caption for Figure 1}
        \label{fig:figure1}
    \end{subfigure}
    \begin{subfigure}[b]{0.475\textwidth}
        \centering
        \includegraphics[width=\textwidth]{fig/tune_subsample.png}
        %\caption{Caption for Figure 2}
        \label{fig:figure2}
    \end{subfigure}
    %\caption{Main caption for the three figures.}
    \label{fig:three_figures}
\end{figure}

\begin{table}[h!]
\centering
\begin{tabular}{c|c|c|c}
\hline
$r^2$ & \textbf{training} & \textbf{validation} & \textbf{testing} \\ \hline
tree      & 1.08\%      & 0.43\%      & 0.96\%       \\ \hline
linear     & 0.81\%      & 0.48\%      & 0.90\%     \\ \hline
\end{tabular}
\end{table}

\end{frame}




\begin{frame}{Simulation}

Strategy:
\begin{itemize}
    \item If predicted move $p>0$
    \item Target position is $\text{tp} = c\cdot p-\text{hbas} - thr$
    \item $\text{tp}\leftarrow \min\left\{\max\left\{\text{tp}, 0, \text{cur\_pos}\right\}, bnd \right\}$
\end{itemize}



Find parameters that maximize performance on the training and validation set
\begin{itemize}
    \item $[\text{trade/day}>1.5]\cdot [\text{PnL/size}>0.3]\cdot \text{sharpe}$
\end{itemize}


\end{frame}

\begin{frame}{Simulation}


\begin{figure}[h!]
    \centering
    \begin{subfigure}[b]{0.32\textwidth}
        \centering
        \includegraphics[width=\textwidth]{fig/simu_1.png}
        %\caption{Caption for Figure 1}
        \label{fig:figure1}
    \end{subfigure}
    \begin{subfigure}[b]{0.32\textwidth}
        \centering
        \includegraphics[width=\textwidth]{fig/simu_2.png}
        %\caption{Caption for Figure 2}
        \label{fig:figure2}
    \end{subfigure}
    \begin{subfigure}[b]{0.32\textwidth}
        \centering
        \includegraphics[width=\textwidth]{fig/simu_3.png}
        %\caption{Caption for Figure 3}
        \label{fig:figure3}
    \end{subfigure}
    %\caption{Main caption for the three figures.}
    \label{fig:three_figures}
\end{figure}


\begin{table}[h!]
\centering
\begin{tabular}{c|c|c|c}
\hline
sharpe & \textbf{training} & \textbf{validation} & \textbf{testing} \\ \hline
tree      & 5.32      & 2.04    & 3.73      \\ \hline
linear     & 3.18      & 1.29      & 2.34     \\ \hline
\end{tabular}
\end{table}


\end{frame}



\begin{frame}{Potential future directions}



\begin{itemize}
    \item Only consider data with emd\_4h$>50\%$, lasso provides larger $r^2$.
    \item Different horizons
    \item Trade $A-\beta B$
\end{itemize}

\end{frame}







\begin{frame}
\begin{center}
\begin{LARGE}
\textcolor{black}{Thanks!}
\end{LARGE}
\end{center}
\end{frame}




\begin{frame}{Predict forward return in 2-3 hours}


\begin{figure}[h!]
    \centering
    \begin{subfigure}[b]{0.32\textwidth}
        \centering
        \includegraphics[width=\textwidth]{fig/simu_tree_2h_1.png}
        %\caption{Caption for Figure 1}
        \label{fig:figure1}
    \end{subfigure}
    \begin{subfigure}[b]{0.32\textwidth}
        \centering
        \includegraphics[width=\textwidth]{fig/simu_tree_2h_2.png}
        %\caption{Caption for Figure 2}
        \label{fig:figure2}
    \end{subfigure}
    \begin{subfigure}[b]{0.32\textwidth}
        \centering
        \includegraphics[width=\textwidth]{fig/simu_tree_2h_3.png}
        %\caption{Caption for Figure 3}
        \label{fig:figure3}
    \end{subfigure}
    %\caption{Main caption for the three figures.}
    \label{fig:three_figures}
\end{figure}


\begin{table}[h!]
\centering
\begin{tabular}{c|c|c|c}
\hline
 & \textbf{training} & \textbf{validation} & \textbf{testing} \\ \hline
$r^2$      & 1.30\%      & 1.07\%     & 1.05\%      \\ \hline
trade per day      & 2.15      & 2.33      & 3.03      \\ \hline
pnl per size      & 2.66      & 1.5      & 1.62      \\ \hline
sharpe      & 4.68      & 3.45      & 3.23     \\ \hline
\end{tabular}
\end{table}


\end{frame}



%\input{newton}

% \section[Outline]{}
% \frame{\tableofcontents}

% \input{content/extra.tex}

\end{document}
